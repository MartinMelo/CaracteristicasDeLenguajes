\documentclass[a4paper,10pt]{article} % a4paper es obvio y 10pt es el tamaño
                                      % de la tipografía básica
\usepackage[utf8]{inputenc} % Permite escribir acentos (leer este archivo en 
                            % UTF8!)
\usepackage[T1]{fontenc}    % Permite que el texto que se produce se pueda
                            % copiar (incluídos los símbolos)
\usepackage[spanish]{babel} % Hace que las palabras claves (como la fecha)
                            % salgan en español
\usepackage{fullpage}       % Pone márgenes más chicos que los estándares

\setlength\parindent{0em}   % Elimina la identación

\usepackage{amsmath}        % Provee el entorno align*

\newcommand\fun{\textsf{fun}~}  % Una especie de macro para usar \fun
\newcommand\fix{\textsf{fix}~}
\newcommand\ifZ{\textsf{ifZ}~}
\newcommand\Then{~\textsf{then}~} % Uso mayúsculas para then y else porque en
\newcommand\Else{~\textsf{else}~} % minúsculas son palabras claves de LaTeX


\title{Ejemplo de \LaTeX\ para las tareas}
\author{Alejandro Díaz-Caro (a.k.a.~Jano)}
\date{\today}  % En vez de \today pueden poner el texto que quieran


\begin{document}
\maketitle   % Pone el título autor y fecha definidos antes

\[
  \textsl{Fact} = \fix f.\fun n.\ifZ n \Then 1 \Else n*(f\ (n-1))
\]


\begin{align*}
  \textsl{Fact}~2
  & = (\fix f.\fun n.\ifZ n \Then 1 \Else n*(f\ (n-1)))~2\\
  & \to (\fun n.\ifZ n\Then 1\Else n*(\textsl{Fact}\ (n-1)))~2\\
  & \to \ifZ 2\Then 1\Else 2*(\textsl{Fact}\ (2-1))\\
  & \to \ifZ 2\Then 1\Else 2*(\textsl{Fact}\ 1)\\
  & \to 2*(\textsl{Fact}\ 1)\\
  & \to 2*((\fix f.\fun n.\ifZ n \Then 1 \Else n*(f\ (n-1)))\ 1)\\
  & \to 2*((\fun n.\ifZ n \Then 1 \Else n*(\textsl{Fact}\ (n-1)))\ 1)\\
  & \to 2*(\ifZ 1 \Then 1 \Else 1*(\textsl{Fact}\ (1-1)))\\
  & \to 2*(\ifZ 1 \Then 1 \Else 1*(\textsl{Fact}\ 0))\\
  & \to 2*(1*(\textsl{Fact}\ 0))\\
  & \to 2*(1*((\fix f.\fun n.\ifZ n \Then 1 \Else n*(f\ (n-1)))\ 0))\\
  & \to 2*(1*((\fun n.\ifZ n \Then 1 \Else n*(\textsl{Fact}\ (n-1)))\ 0))\\
  & \to 2*(1*(\ifZ 0 \Then 1 \Else 0*(\textsl{Fact}\ (0-1))))\\
  & \to 2*(1*1)\\
  & \to 2*1\\
  & \to\fbox{2}\\
\end{align*}
\end{document}



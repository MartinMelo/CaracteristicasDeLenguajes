\documentclass[10pt]{article} % a4paper es obvio y 10pt es el tama�o
                                      % de la tipograf�a b�sica
\usepackage[paperwidth=10in, paperheight=8.5in]{geometry}
\usepackage[utf8]{inputenc} % Permite escribir acentos (leer este archivo en 
                            % UTF8!)
\usepackage[T1]{fontenc}    % Permite que el texto que se produce se pueda
                            % copiar (inclu�dos los s�mbolos)
\usepackage[spanish]{babel} % Hace que las palabras claves (como la fecha)
                            % salgan en espa�ol
\usepackage{fullpage}       % Pone m�rgenes m�s chicos que los est�ndares

\setlength\parindent{0em}   % Elimina la identaci�n

\usepackage{amsmath,amssymb,graphicx,stmaryrd}        

\usepackage{proof}        % Para escribir el infer y hacer la linea para la definicion inductiva*

\newcommand\fun{\textsf{fun}~}  % Una especie de macro para usar \fun
\newcommand\fix{\textsf{fix}~}
\newcommand\Let{\textsf{Let}~}
\newcommand\In{\textsf{in}~}
\newcommand\ifZ{\textsf{ifZ}~}
\newcommand\Nat{\textsf{Nat}~}
\newcommand\CL{\textsf{CL}~}
\newcommand\CR{\textsf{CR}~}
\newcommand\Node{\textsf{Node}~}
\newcommand\Leaf{\textsf{Leaf}~}
\newcommand\Tree{\textsf{Tree}~}
\newcommand\NoLeaf{\textsf{NoLeaf}~}
\newcommand\ifLeaf{\textsf{ifLeaf}~}
\newcommand\Then{~\textsf{then}~} % Uso may�sculas para then y else porque en
\newcommand\Else{~\textsf{else}~} % min�sculas son palabras claves de LaTeX


\title{Entrega 9}
\author{Martin Alejandro Melo}
\date{\today}


\begin{document}
\maketitle  
	
	
		\textbf{Extender semantica de arboles binarios.}\\\\
		
    $ \llbracket \Tree t~r~l \rrbracket _\Gamma = $
			\[ \left\{ 
			\begin{array}{ll}
						\mbox{ $ \llbracket \Tree A \rrbracket _\Gamma ~~~~~si~~~~ \llbracket t \rrbracket _\Gamma = A ~y~ \llbracket u \rrbracket _\Gamma = \Tree A ~y~ \llbracket v \rrbracket _\Gamma = \Tree A  $}\\
						\mbox{ $ \text{Error} ~~~~~si~~~~ \llbracket t \rrbracket _\Gamma = \text{Error} ~o~ \llbracket u \rrbracket _\Gamma = \text{Error} ~o~ \llbracket v \rrbracket _\Gamma = \text{Error}  $}\\
						\mbox{ $ \bot _\Tree _A  ~~~~~ \text{si no se cumple lo anterior}$}
					\end{array}
			\right. \] 
		\\\\
		$ \llbracket \Leaf t \rrbracket _\Gamma = $
			\[ \left\{ 
			\begin{array}{ll}
						\mbox{ $ \llbracket \Tree A \rrbracket _\Gamma ~~~~~si~~~~ \llbracket t \rrbracket _\Gamma = A $}\\
						\mbox{ $ \text{Error}  ~~~~~si~~~~ \llbracket t \rrbracket _\Gamma = A $}\\
						\mbox{ $ \bot _\Tree _A  ~~~~~ \text{si no se cumple lo anterior} $}
					\end{array}
			\right. \] 
		\\\\
		$ \llbracket \ifLeaf t \Then u \Else v \rrbracket _\Gamma = $
			\[ \left\{ 
			\begin{array}{ll}
						\mbox{ $ \llbracket u \rrbracket _\Gamma ~~~~~si~~~~ \llbracket t \rrbracket _\Gamma = Leaf A $}\\
						\mbox{ $ \llbracket v \rrbracket _\Gamma ~~~~~si~~~~ \llbracket t \rrbracket _\Gamma = Tree A $}
					\end{array}
			\right. \] 
		\\\\
    $ \llbracket \Node t~r~l \rrbracket _\Gamma  = \llbracket t \rrbracket _\Gamma $\\\\
    $ \llbracket \CL t~r~l \rrbracket _\Gamma = \llbracket l \rrbracket _\Gamma $\\\\
    $ \llbracket \CR t~r~l \rrbracket _\Gamma = \llbracket r \rrbracket _\Gamma $



	\end{document}

\documentclass[a4paper,10pt]{article} % a4paper es obvio y 10pt es el tamaño
                                      % de la tipografía básica
\usepackage[utf8]{inputenc} % Permite escribir acentos (leer este archivo en 
                            % UTF8!)
\usepackage[T1]{fontenc}    % Permite que el texto que se produce se pueda
                            % copiar (incluídos los símbolos)
\usepackage[spanish]{babel} % Hace que las palabras claves (como la fecha)
                            % salgan en español
\usepackage{fullpage}       % Pone márgenes más chicos que los estándares

\setlength\parindent{0em}   % Elimina la identación

\usepackage{amsmath,amssymb,graphicx}        

\usepackage{proof}        % Para escribir el infer y hacer la linea para la definicion inductiva*

\newcommand\fun{\textsf{fun}~}  % Una especie de macro para usar \fun
\newcommand\fix{\textsf{fix}~}
\newcommand\Let{\textsf{Let}~}
\newcommand\In{\textsf{in}~}
\newcommand\ifZ{\textsf{ifZ}~}
\newcommand\Nat{\textsf{Nat}~}
\newcommand\CL{\textsf{CL}~}
\newcommand\CR{\textsf{CR}~}
\newcommand\Node{\textsf{Node}~}
\newcommand\Leaf{\textsf{Leaf}~}
\newcommand\Tree{\textsf{Tree}~}
\newcommand\NoLeaf{\textsf{NoLeaf}~}
\newcommand\ifLeaf{\textsf{ifLeaf}~}
\newcommand\Then{~\textsf{then}~} % Uso mayúsculas para then y else porque en
\newcommand\Else{~\textsf{else}~} % minúsculas son palabras claves de LaTeX


\title{Entrega 4 - Primera Parte}
\author{Martin Alejandro Melo}
\date{\today}


\begin{document}
\maketitle  
	
	
	\textbf{Derivar tipo con Hindley y Robinson a: Node ((fun x.leaf x) (fun y.y)).}
	\vspace{5mm}
	
	\infer{\Gamma \vdash \Node ((\fun x.\Leaf x) (\fun y.y)) \leadsto C, \{B\Rightarrow \Tree B = (A\Rightarrow A)\Rightarrow X , X= \Tree C \} }
		{	
			\infer{\Gamma \vdash (\fun x.\Leaf x) (\fun y.y)\leadsto X, \{B\Rightarrow \Tree B = (A\Rightarrow A)\Rightarrow X\}}
			{
			 \infer{\Gamma \vdash \fun x.\Leaf x \leadsto B \Rightarrow \Tree B, \emptyset}
				{
					\infer{x:B\vdash x \leadsto B, \emptyset}{}
				}
				& 
				\infer{\Gamma \vdash \fun y.y \leadsto A\Rightarrow A, \emptyset}
				{
					\infer{y:A\vdash y \leadsto A, \emptyset}{}
				}
			}	
		}
		\vspace{5mm}
		
		\textbf{Robinson}
		
		 
		\[ \left\{ 
		\begin{array}{ll}
					\mbox{ $ B\Rightarrow \Tree B = (A\Rightarrow A)\Rightarrow X $ }\\
					\mbox{X= \Tree C}
				\end{array}
		\right. \] 		
		$ \Rightarrow	(1)$
		\[ \left\{ 
		\begin{array}{ll}
					\mbox{ $ B = (A\Rightarrow A) $ }\\
					\mbox{ $ \Tree B = X $ }\\
					\mbox{X= \Tree C}
				\end{array}
		\right. \]
		$ \Rightarrow	(1)$
		\[ \left\{ 
		\begin{array}{ll}
					\mbox{ $ B = (A\Rightarrow A) $ }\\
					\mbox{ $ \Tree B = \Tree B $ }\\
					\mbox{ $ \Tree B = \Tree C $}
				\end{array}
		\right. \]
		$ \Rightarrow	(4)$
		\[ \left\{ 
		\begin{array}{ll}
					\mbox{ $ B = (A\Rightarrow A) $ }\\
					\mbox{ $ \Tree B = \Tree C $}\\
					\mbox{ $ B = C $}
				\end{array}
		\right. \]
		$ \Rightarrow	(1)$
		\[ \left\{ 
		\begin{array}{ll}
					\mbox{ $ (A\Rightarrow A) = (A\Rightarrow A) $ }\\
					\mbox{ $ \Tree (A\Rightarrow A) = \Tree C $}\\
					\mbox{ $ (A\Rightarrow A) = C $}
				\end{array}
		\right. \]
		
		$ S = [(A\Rightarrow A)/C] $ \\
		El tipo es: $ (A\Rightarrow A) $
		
	\end{document}

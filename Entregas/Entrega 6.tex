\documentclass[10pt]{article} % a4paper es obvio y 10pt es el tama�o
                                      % de la tipograf�a b�sica
\usepackage[paperwidth=10in, paperheight=8.5in]{geometry}
\usepackage[utf8]{inputenc} % Permite escribir acentos (leer este archivo en 
                            % UTF8!)
\usepackage[T1]{fontenc}    % Permite que el texto que se produce se pueda
                            % copiar (inclu�dos los s�mbolos)
\usepackage[spanish]{babel} % Hace que las palabras claves (como la fecha)
                            % salgan en espa�ol
\usepackage{fullpage}       % Pone m�rgenes m�s chicos que los est�ndares

\setlength\parindent{0em}   % Elimina la identaci�n

\usepackage{amsmath,amssymb,graphicx}        

\usepackage{proof}        % Para escribir el infer y hacer la linea para la definicion inductiva*

\newcommand\fun{\textsf{fun}~}  % Una especie de macro para usar \fun
\newcommand\fix{\textsf{fix}~}
\newcommand\Let{\textsf{Let}~}
\newcommand\In{\textsf{in}~}
\newcommand\ifZ{\textsf{ifZ}~}
\newcommand\Nat{\textsf{Nat}~}
\newcommand\CL{\textsf{CL}~}
\newcommand\CR{\textsf{CR}~}
\newcommand\Node{\textsf{Node}~}
\newcommand\Leaf{\textsf{Leaf}~}
\newcommand\Tree{\textsf{Tree}~}
\newcommand\NoLeaf{\textsf{NoLeaf}~}
\newcommand\ifLeaf{\textsf{ifLeaf}~}
\newcommand\Then{~\textsf{then}~} % Uso may�sculas para then y else porque en
\newcommand\Else{~\textsf{else}~} % min�sculas son palabras claves de LaTeX


\title{Entrega 6}
\author{Martin Alejandro Melo}
\date{\today}


\begin{document}
\maketitle  
	
	

\[
   (\fun x.x+x) ((\fun y.y)3)
\]
\textsl{CBN}\\
\begin{align*}
\infer{\Gamma(\fun x.x+x) ((\fun y.y)3) \hookrightarrow 6}
	{
		\infer{\Gamma\vdash \fun x.x+x \hookrightarrow <x,x+x,\varnothing>}{} 
		&
		\infer{x=<((\fun y.y)3),\Gamma>\vdash x+x \hookrightarrow 6}
		{
			\infer{x=<((\fun y.y)3),\Gamma> \vdash x \hookrightarrow 3}
			{
				\infer{\vdash(\fun y.y) 3 \hookrightarrow 3}
				{
					\infer{\vdash \fun y.y \hookrightarrow <y,y,\varnothing>}{}
					& 
					\infer{y=<3,\varnothing>\vdash 3 \hookrightarrow 3}
					{
						\infer{\vdash 3 \hookrightarrow 3}{}
					}
				}
			}
			&
			\infer{x=<((\fun y.y)3),\Gamma> \vdash x \hookrightarrow 3}
			{
				\infer{\vdash(\fun y.y) 3 \hookrightarrow 3}
				{
					\infer{\vdash \fun y.y \hookrightarrow <y,y,\varnothing>}{}
					& 
					\infer{y=<3,\varnothing>\vdash 3 \hookrightarrow 3}
					{
						\infer{\vdash 3 \hookrightarrow 3}{}
					}
				}
			}
		}
	}
\end{align*}

\textsl{CBV}\\
\begin{align*}
\infer{\Gamma(\fun x.x+x) ((\fun y.y)3) \hookrightarrow 6}
	{
		\infer{\Gamma\vdash \fun x.x+x \hookrightarrow <x,x+x,\varnothing>}{} 
		&
		\infer{x=3\vdash x+x \hookrightarrow 6}
		{
			\infer{x=3\vdash x \hookrightarrow 3}
			{
				\infer{\vdash 3 \hookrightarrow 3}{}
			}
			&
			\infer{x=3\vdash x \hookrightarrow 3}
			{
				\infer{\vdash 3 \hookrightarrow 3}{}
			}
		} 
		&
		\infer{\Gamma\vdash(\fun y.y) 3 \hookrightarrow 3}
				{
					\infer{\Gamma\vdash \fun y.y \hookrightarrow <y,y,\varnothing>}{}
					& 
					\infer{y=<3,\varnothing>\vdash y \hookrightarrow 3}
					{
						\infer{\vdash 3 \hookrightarrow 3}{}
					}
				}
	}
\end{align*}

	\end{document}

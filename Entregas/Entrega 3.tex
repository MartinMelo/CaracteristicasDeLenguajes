\documentclass[a4paper,10pt]{article} % a4paper es obvio y 10pt es el tamaño
                                      % de la tipografía básica
\usepackage[utf8]{inputenc} % Permite escribir acentos (leer este archivo en 
                            % UTF8!)
\usepackage[T1]{fontenc}    % Permite que el texto que se produce se pueda
                            % copiar (incluídos los símbolos)
\usepackage[spanish]{babel} % Hace que las palabras claves (como la fecha)
                            % salgan en español
\usepackage{fullpage}       % Pone márgenes más chicos que los estándares

\setlength\parindent{0em}   % Elimina la identación

\usepackage{amsmath,amssymb,graphicx}        

\usepackage{proof}        % Para escribir el infer y hacer la linea para la definicion inductiva*

\newcommand\fun{\textsf{fun}~}  % Una especie de macro para usar \fun
\newcommand\fix{\textsf{fix}~}
\newcommand\ifZ{\textsf{ifZ}~}
\newcommand\Nat{\textsf{Nat}~}
\newcommand\CL{\textsf{CL}~}
\newcommand\CR{\textsf{CR}~}
\newcommand\Node{\textsf{Node}~}
\newcommand\Leaf{\textsf{Leaf}~}
\newcommand\Tree{\textsf{Tree}~}
\newcommand\NoLeaf{\textsf{NoLeaf}~}
\newcommand\ifLeaf{\textsf{ifLeaf}~}
\newcommand\Then{~\textsf{then}~} % Uso mayúsculas para then y else porque en
\newcommand\Else{~\textsf{else}~} % minúsculas son palabras claves de LaTeX


\title{Entrega 3}
\author{Martin Alejandro Melo}
\date{\today}


\begin{document}
\maketitle   % Pone el título autor y fecha definidos antes

	\textsl{Extender Algoritmo de Hindley y Robinson para Arboles Binarios y derivar el tipo para el siguiente termino:} \\
	
	
	\text{\CL ( (\fun x.x) (\Leaf 2))} \\
	
	
	\textbf{Para Hindley:} \\
	
	
	\infer{\Gamma \vdash \Leaf t \leadsto A \rightarrow \Tree A,\phi  }{\Gamma \vdash t \leadsto A,\phi }
	\vspace{5mm}
	\infer{\Gamma \vdash \CL t \leadsto Tree A \rightarrow \Tree A,\Theta  }{\Gamma \vdash t \leadsto \NoLeaf A,\Theta }
	\vspace{5mm}
	\infer{\Gamma \vdash \CR t \leadsto Tree A \rightarrow \Tree A,\Theta  }{\Gamma \vdash t \leadsto \NoLeaf A,\Theta }
	\vspace{5mm}

	
	\textbf{Para Robinson:} \\
	(1) Cada vez que aparece Tree A => NoLeaf A, Dar error. \\
	(2) Cada vez que aparece NoLeaf A, Se puede reemplazar por Tree A. \\
	\end{document}


